\section{星云链基础服务及开发工具}
\subsection{域名服务}

在区块链的世界中,应用都以账号为中心,由于区块链的匿名性,账号地址是非常长的无意义的字符串,并不是用户友好的。用户可能容易出错,意外地投入资金导致丢失,或者与错误对象互动。如果用户能够使用容易记忆的域名,将是巨大的优势。星云链团队在区块链上用智能合约实现一个类似DNS的域名系统,称为NNS(Nebulas Name Service),保证其自由、免费、开放。任何第三方开发者可以独立地或者基于NNS实现自己的域名解析服务。

例如,alice为她的账号地址0x5d65d971895edc438f465c17db6992698a52318d5c17db69申请了域名alice.nns,Bob如果想转账给alice,在收款人信息中只要填写alice.nns即可,钱包会通过NNS服务,把资金转账到正确的收款人地址。

NNS的申请规则如下:
\begin{itemize}
	\item 一级子域名保留,不可申请,例如*.nns,*.com,*.org,*.edu等等,用户只能申请二级子域名。
	\item 所有参与ICO项目的用户,在星云链上线以后,可以免费获得赠送一个二级子域名。用户向星云链开发组提出域名申请,只要和其他人的不冲突,可以直接在系统内获得赠送,永久保留。
	\item NNS服务开放后,用户可以查询域名是否已经被占用,如果未被占用,可以通过智能合约,提出竞价。竞价是公开的,任何人都可以查询其他人的竞价并且随时更新自己的竞价。
	\item 竞价期结束以后,价高者得此域名,智能合约会锁定用户的竞价资金。域名有效期为一年。一年后用户自由决定是否续约,同意续约后,有效期再延长一年。如果不续约,竞价资金自动退回用户账户,域名数据清空,重新变为可用。
	\item 用户可以随时放弃域名所有权,竞价资金自动退回用户账户,域名数据清空,重新变为可用。
	\item 用户可以无偿或者有偿转让域名所有权,域名之间的交易,星云链团队不做任何干预。
\end{itemize}


\subsection{闪电网络}
当前所有的公有链网络都面临系统的扩展性问题,例如比特币网络每秒只能处理7个交易,以太坊每秒处理15个交易。引入类PoS的共识算法,可以避免挖矿计算,大大加快共识速度,但是面对现实世界的海量微支付场景,公有链仍然面临极大的挑战。闪电网络\cite{poon2015bitcoin}于2015年2月被提出,设计思路是为交易双方建立一个微支付通道网络,双方大量的支付都可以在链外直接多次、高频、双向地通过轧差方式实现瞬间确认。当交易结果需要结算时,再把最终结果提交到区块链确认,理论上可以实现每秒百万笔的转账。双方若无直接的点对点支付通道,只要网络中存在一条连通双方的、由多个支付通道构成的支付路径,也可以利用这条支付路径实现资金在双方之间的可靠转移。闪电网络在比特币和以太坊上都已经有概念性的验证。

星云链把闪电网络当做区块链的基础设施予以实现,并且提供足够的灵活性设计。任何第三方开发者,都可以在星云链上利用闪电网络的基础服务,做高频交易场景的应用开发。星云链也将发布世界首款支持闪电网络的钱包App。


\subsection{搜索引擎}
随着越来越多的智能合约被开发者部署,用户面对海量的智能合约,搜索的需求也变得突出起来。智能合约仅仅只有代码,并不包含任何功能描述,很难利用搜索引擎技术为智能合约建立索引。我们为智能合约建立规范,任何符合此规范的智能合约,都能被检索并被用户搜索到,鼓励开发者在创建智能合约时,提供合约元信息。

\begin{lstlisting}[frame=single]
contract SearchableContract {
   string public language;
   string public author;
   string public name;
   string public title;
   string public description;
   string public tags;
}
\end{lstlisting}

星云链开发组实现一个搜索服务,实时检索所有实现此规范的合约,做多语言的分词,建立全文索引。排序规则考虑了多个因素,例如匹配字段,文本相关度,合约的NR Rank值,合约的交易数量和频度,与合约发生交易的用户NR Rank值等等,最终提供一个友好的Web界面给用户使用,根据用户的反馈不断优化排序算法。所有搜索后台代码都将开源给社区。第三方开发者也可据此创建独立的搜索服务。

\subsection{开发者工具}
完善的开发者工具,对于区块链应用开发者来说是非常重要的。目前各种公有区块链的开发者工具链都不够完善,对于大多数开发者是很大的阻碍。星云链开发组将提供丰富的开发者工具,包括独立的智能合约开发IDE,区块浏览器,各种流行IDE的插件支持(包括eclipse,JetBrains,Visual Studio,Sublime,VIM,Atom等),调试器,模拟器,智能合约形式化验证工具,各种高级语言的后台SDK,移动端SDK等。
