\section{智能合约}

\subsection{图灵完备的智能合约编程语言}
智能合约是一套以数字形式定义的承诺(promises),包括合约参与方可以在上面执行这些承诺的协议。在物理上,智能合约的载体是计算机可识别并运行的计算机代码。比特币脚本语言是一种命令式的、基于栈的编程语言,由于它是非图灵完备的,所以应用上有一定的局限性。以太坊是全世界第一个实现图灵完备的智能合约的区块链系统,编程语言是Solidity、Surpent,使得应用开发者们可以高效快速地开发各式各样的应用程序。智能合约代码发布到区块链上之后,能够无需中介的参与,在区块链上自动执行。星云链中的智能合约编程语言,在初期时完全兼容以太坊的Solidity,方便开发者为以太坊开发的智能合约应用无缝的迁移到星云链中来。我们在Solidity语言中增加一些跟Nebulas Rank相关的指令集,方便开发者获取任意用户的NR值。后续我们会设计自己的智能合约二进制规范,推出各种编程语言的支持,使得开发者可以用自己喜欢的高级语言编程,例如Java、Python、Go、JavaScript、Scala等。

\subsection{合约可升级设计}
目前以太坊智能合约的设计是代码一经部署,不可变化,代码逻辑从部署的时刻起,便永远不再具有升级的能力。智能合约如果作为协议来看,不可变化是其要求的,代表着一种协议的约定,运行行为都是确定性的。但是随着智能合约开始获得越来越多的使用,其流程和代码也变得越来越复杂,人们发现,就像现实世界的合同一样,如果没有认真审核的话,在设计和编码过程中难以避免人工失误的产生,一旦被黑客找到漏洞,损失往往是巨大的。2016年6月,The DAO攻击事件,由于一个代码缺陷,导致以太坊用户损失了共计6000万美元的损失;最近Parity钱包的漏洞,导致15万个以太币的流失,价值3000万美元。比特币由于其设计上的非图灵完备性,删减了许多脚本指令,所以其安全性是极高的。

虽然目前有各种智能合约编程的最佳实践,以及更严格的审核流程,甚至出现形式化验证工具,通过数学证明的方式验证智能合约的确定性。但是既然是代码,就不可能没有漏洞。回顾我们现在的中心化的互联网世界,各种互联网服务都是可以升级的,弥补开发过程中发生的各种漏洞。任何一个完美的应用系统,都是演化而不是设计出来的。我们认为,解决智能合约安全性的根本问题,需要有一个好的智能合约可升级设计方案。

以太坊上的智能合约可升级设计有一些解决方案,大体上分为两类:一类是对外公开proxy contract(代理合约),代理合约的代码非常简单,仅仅把请求转发给后面的真正的功能合约。当需要升级合约时,把代理合约的内部功能合约指针指向新的合约即可;第二类是把合约的代码和存储分离,存储合约负责提供方法,供外部合约读写内部状态,代码合约做真正的业务逻辑,升级时只需要部署新的代码合约,不丢失所有的状态。这两类方案都有其局限性,不能解决所有问题:合约的代码和存储分离在设计上增加了很多复杂度,有时候甚至不可行;代理合约虽然能够指向新合约,但是老合约的状态数据并不能迁移;有些合约在开始开发时,没有良好的设计,没有为以后的升级留下接口。

我们设计一种简洁的智能合约升级方案:在语言层面上,我们支持一个合约的状态变量供另外一个合约直接读写(符合安全约束)。这是一个例子,假如有个Token合约,代码如下:
\begin{lstlisting}[frame=single]
contract Token {
  mapping (address => uint256) balances shared;

  function transfer(address _to, uint256 _value) returns (bool success) {
     if (balances[msg.sender] >= _value) {
       balances[msg.sender] -= _value;
       balances[_to] += _value;
       return true;
     } else {
       return false;
     }
   }
   function balanceOf(address _owner) constant returns (uint256 balance) {
       return balances[_owner];
   }
}
\end{lstlisting}

合约部署时,balances变量用关键字shared标识,编译成字节码运行时,虚拟机会为该变量单独设计存储区域。不用关键字shared声明的变量,都不可以被其它合约直接访问。

假如原代码的transfer函数需要修改一个bug,对\_value做检查,部署新的智能合约代码:

\begin{lstlisting}[frame=single]
[baseContractAddress="0x5d65d971895edc438f465c17db6992698a52318d"]
//baseContractAddress是老合约的地址
contract Token {
  mapping (address => uint256) balances shared;

  function transfer(address _to, uint256 _value) returns (bool success) {
     if (balances[msg.sender] >= _value ^&& _value > 0^) {
       balances[msg.sender] -= _value;
       balances[_to] += _value;
       return true;
     } else {
       return false;
     }
   }
   function balanceOf(address _owner) constant returns (uint256 balance) {
       return balances[_owner];
   }
}
\end{lstlisting}

新的合约部署以后,老的合约可以选择selfdestruct,不能再被访问,但是shared变量依然被永久保留。新的合约可以完全继承老合约的balances资产,全部的状态都不丢失,不需要做额外的迁移工作。但是在开发智能合约时,对关键的状态变量声明为shared是必须的,编译器会对变量的存储区域做特殊处理,保证其可以被其它授权的合约访问。

为了保证安全,升级合约和老合约必须是相同的creator,否则运行时会抛异常。

这种设计存在道德上的问题,因为合约的内容条款一旦拟订,其实是不应该被修改的,至少修改必须征询合约受众的同意。我们计划引入投票机制,以批准智能合约的升级,而不是默默的被合约创建者修改。

通过这种可升级方案,The DAO或者Parity类似的漏洞攻击事件,可以更快的被修复,而不是通过硬分叉的方式。并且修复以后,所有用户的资产都不需要迁移,仍然继续使用。


