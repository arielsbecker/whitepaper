\section{简介}

\subsection{区块链技术简介}
区块链技术来源于比特币,比特币的概念最初由中本聪在2008年提出,是一种去中心化的数字货币。比特币不依靠特定货币机构发行,而是根据特定算法,依靠大量计算产生。产生的过程实质上是用计算机解决一项复杂的数学问题,来保证比特币网络分布式记账系统的一致性。区块链是以比特币为代表的数字加密货币体系的核心支撑技术,核心优势是去中心化,能够通过运用数据加密、时间戳、分布式共识和经济激励等手段,在节点无需互相信任的分布式系统中实现去中心化信用的点对点交易、协调与协作,从而为解决中心化机构普遍存在的高成本、低效率和数据存储不安全等问题提供了解决方案。需要说明的是,区块链技术本身不是一个全新的技术创新,而是作为一系列技术的组合(包括点对点通讯,密码学,块链数据结构等)产生的模式创新。

\subsection{商业和技术挑战}
区块链技术为代表的去中心化,自主治理的系统,正在引起越来越多人的重视和研究。当前全球区块链项目已经超过2000个,全球加密数字资产总体价值达到900亿美元。区块链\&数字资产领域的用户人群也正在快速增加。从2013年初的全球200万用户,到2017年初的2000万用户。我们认为,在2020年左右,全球区块链/数字资产用户会达到或超过2亿。在2025年前后,全球用户有望达到10亿规模。在这种规模下,区块链技术面临诸多的挑战,比如区块链信息的交互、协议的升级,以及区块链应用\&智能合约的有效及相关性判断问题。

\textbf{区块链信息的交互},即互操作性不足的问题,行业内已经有很多共识,迄今为止还没有完善的解决方案。主要是关于链内及链外,同时也包括“跨链”的数据及资产交互问题,这极大程度的限制了区块链的发挥空间,形成所谓区块链内外的“信息孤岛”。我们也看到,越来越多的区块链项目尝试对此有所突破。同时,对于不同区块链上的信息\&资产当前缺少统一的协议和标准,导致各区块链系统内外部的互操作性严重不足。

\textbf{区块链系统协议的升级},不同于普通软件的版本迭代,往往会引发区块链“硬分叉”或“软分叉”。以比特币为例,社区关于区块扩容至今仍然存在巨大的争议,导致比特币协议进化缓慢,区块容量严重不足,出现过近100万笔交易在交易池等待被写入区块。目前来看比特币转账时间存在很大不确定性,用户很多时候不得不额外支付高昂的“交易加速费”,严重损害体验。另外,从以太坊的“硬分叉”来看,虽然暂时解决了问题,但是产生了ETH和ETC“重资产”和社区分裂的“副作用”。

\textbf{区块链应用及智能合约的高效检索},随着区块链上各种应用 (DApp)的快速增长,逐 渐成为一个巨大的挑战。以以太坊为例,基于以太坊的DApp总数已经远超数百个,试想如果区块链世界DApp接近App Store里应用总量规模的话,如何发现并找到好用的DApp就是个很大问题。从这里来看,当前区块链世界缺乏更多维度的信息和价值评判尺度,用来帮助用户更好的探索区块链世界。

\subsection{星云链设计原则}
面对上述机遇和挑战,我们要设计一个基于价值激励的自进化区块链系统。具体来说,我们有以下设计原则:
\begin{itemize}
	\item 完全兼容以太坊

	以太坊已经有巨大的生态,是一个非常成功的公有区块链平台。星云链希望尽可能的借鉴以太坊优秀的设计,从协议底层以及智能合约编程上完全兼容以太坊,使得基于以太坊开发的产品能够零成本的迁移到星云链上。
	\item 公正的排序算法,定义价值尺度

	我们认为,区块链世界需要一个普适的价值尺度,用于衡量区块链底层简单数据的价值,发现信息的更高维度,从而探索并挖掘区块链世界的更大价值。类似Google的PageRank,我们也提出区块链世界的NR(Nebulas Rank)算法,综合考虑了区块链上的资金流动性,以及资金传播的速度、广度和深度,给区块链用户做公正的排序。NR是星云链赋予区块链世界的价值尺度,用来衡量区块链中各个用户、智能合约、DApp的重要性。NR有巨大的商业潜力,可以用在搜索、推荐、广告等领域当中。
	\item 更快更合理的共识算法

区块链的分布式交易总账需要在尽可能短的时间内做到安全、明确及不可逆,需要一个坚实的共识算法。比特币、以太坊的PoW(Proof of Work)共识算法,需要消耗大量的计算资源,影响共识速度;Pos(Proof of Stake)股权证明共识,根据用户的资金量来随机选择记账人,共识速度大大加快,但是有nothing-at-stake[ref]的问题,夸大了资本对记账权概率分配的影响,大资本更容易占据生态的话语权,不利于公链生态的自由发展。其它还有各种共识算法,都有一些弊端。

星云链基于NR,不是把用户的资产作为记账权,而是把用户的价值尺度作为记账评判标准,对星云链生态系统贡献较大的用户,赋予记账资格,提出了基于账户声望的PoR算法,融合PoS中的经济惩罚,遏制了PoS的垄断性,为生态自由发展助力。

	\item 核心协议的可升级,实现自我进化

我们认为,一个良态的区块链系统应该能够实现自我进化。在较少外部干涉的情况下,实现更快的计算、更强的系统、及更好的体验。星云链在区块结构上进行更好的设计,使得核心协议的升级能够通过区块数据的更新实现,不需要代价高昂的软、硬分叉。星云链具备自我升级进化的设计,未来比其它的公有链具有更快的发展速度,更大的生存潜力。

	\item 对区块链生态的有效激励

在星云链中,我们提出面向智能合约和DApp开发者的DIP(Developer Incentive Protocol 开发者激励协议)。DIP的核心思想是对最近一段时间周期内新部署上线,NR值上涨较快的智能合约或DApp的开发者,给予他们相应的开发者激励。激励由记账人负责写入区块。

普通公有链系统仅仅对记账人(即矿工)奖励,星云链不仅奖励矿工,还奖励优秀的智能合约开发者。借助于DIP的正向反馈机制,更多的开发者将被持续激励创造更高价值的智能合约和DApp,从而构建面向开发者社区的正向反馈生态。

	\item 图灵完备可升级的智能合约

目前公有链的智能合约不具有升级的能力,如果开发过程中没有认真审核的话,很难避免人工失误,一旦被黑客找到漏洞,损失往往是巨大的。2016年6月,The DAO攻击事件,由于一个代码缺陷,导致以太坊用户损失了共计6000万美元的损失;最近Parity钱包的漏洞,导致15万个以太币的流失,价值3000万美元。以太坊由于The DAO的攻击,不得不进行硬分叉,造成很多诟病:一个设计不好的智能合约,绑架了整个区块链生态。星云链希望从底层提供智能合约升级的能力,面对漏洞,能够更快的响应和升级,再也不需要区块链强行分叉来应对黑客事件。

\end{itemize}

