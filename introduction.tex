\section{简介}

\subsection{区块链技术简介}
区块链技术来源于比特币Bitcoin~\cite{Nakamoto2008},其概念由中本聪在2008年提出,是一种去中心化的数字货币,其产生不依赖于任何货币发行机构,而是根据特定算法,依靠大量计算产生,保证比特币网络分布式记账系统的一致性。以太坊Ethereum~\cite{buterin2014ethereum}更进一步,为我们提供了一个通用框架,可以在区块链上运行任何类型的代码,带来了去中心化的应用。区块链是以比特币和以太坊为代表的数字加密货币体系的核心支撑技术,能够通过运用数据加密、时间戳、分布式共识和经济激励等手段,在节点无需互相信任的分布式系统中实现去中心化信用的点对点交易、协调与协作,从而解决中心化机构普遍存在的高成本、低效率和数据存储不安全等问题。

需要说明的是,区块链技术本身不是一个全新的技术创新,而是作为一系列技术的组合(包括点对点通讯,密码学,块链数据结构等)产生的模式创新。

\subsection{商业和技术挑战}
区块链技术为代表的去中心化,自主治理的系统,正在引起越来越多人的重视和研究。当前全球区块链项目已经超过2000个,全球加密数字资产总体价值达到900亿美元。区块链/数字资产领域的用户人群也正在快速增加。从2013年初的全球200万用户,到2017年初的2000万用户。我们认为,在2020年左右,全球区块链/数字资产用户会达到或超过2亿。在2025年前后,全球用户有望达到10亿规模。

随着区块链技术的普及,越来越多区块链技术之上的应用场景被挖掘出来。区块链技术的应用场景已经从最初的数字化货币本身逐步扩展到更多的场景及用户群体中。例如,以以太坊为代表的社区在区块链技术中引入智能合约的概念,Ripple为代表的结算系统。随着应用场景的多样化,用户对区块链技术的诉求也日益增加,我们已经看到很多挑战。

\paragraph{缺失价值尺度}当前区块链世界缺乏更多维度的信息和价值评判尺度,用来帮助用户更好的探索区块链世界。我们认为,区块链世界需要一个普适的价值尺度,来衡量用户和智能合约的价值,上层应用可以在这个普适的价值尺度上结合自身场景挖掘更深层次的价值,这将带来更多的商业模式的创新,就像Google在互联网世界的兴起一样。

\paragraph{区块链系统的升级}不同于普通软件的版本迭代,区块链系统由于其天生的去中心化特性,无法强制用户升级其客户端及协议。因此,区块链系统中的协议升级往往会引发区块链“硬分叉”或“软分叉”,从而造成巨大的损失,这更进一步限制了区块链系统的应用场景。以比特币为例,社区关于区块扩容至今仍然存在巨大的争议,导致比特币协议进化缓慢,区块容量严重不足,出现过近100万笔交易在交易池等待被写入区块。目前来看比特币转账时间存在很大不确定性,用户很多时候不得不额外支付高昂的“交易加速费”,严重损害体验。另外,从以太坊的“硬分叉”来看,虽然暂时解决了问题,但是产生了ETH和ETC“重资产”和社区分裂的“副作用”。

\paragraph{区块链应用生态环境的建立}随着区块链上各种应用(DApp)的快速增长,良好的生态环境是提高用户体验的根本所在。这包括用户如何在海量的区块链应用中检索自己期望的DApp,如何激励开发人员为用户提供更多的DApp,以及如何帮助开发人员更快的构建更好的DApp。以以太坊为例,基于以太坊的DApp总数已经数十万个,试想如果区块链世界中的DApp接近App Store里应用总量规模的话,如何发现并找到用户期望的DApp就是个很大问题。


\subsection{星云链设计原则}
面对上述机遇和挑战,我们要设计一个基于价值激励的自进化区块链系统。具体来说,我们有以下设计原则:
\begin{itemize}
	\item 公正的排序算法,定义价值尺度

我们认为,区块链世界需要一个普适的价值尺度,用于衡量区块链底层简单数据的价值,发现信息的更高维度,从而探索并挖掘区块链世界的更大价值。类似Google的PageRank\cite{Brin2010}\cite{page1999pagerank},我们也提出区块链世界的NR(Nebulas Rank)(见\refsec{sec:rank})算法,综合考虑了区块链上的资金流动性,以及资金传播的速度、广度和深度,给区块链用户做公正的排序。NR是星云链赋予区块链世界的价值尺度,用来帮助开发者结合自身场景有效衡量区块链中各个用户、智能合约、DApp的重要性。NR有巨大的商业潜力,可以用在搜索、推荐、广告等领域当中。

  \item 区块链系统及应用的自我进化

    我们认为,一个良态的区块链系统及其上的应用应该能够实现自我进化。在较少外部干涉的情况下,实现更快的计算、更强的系统、及更好的体验。我们将这种自我进化的能力称之为“Nebulas
    Force
    星云原力”。在星云链的系统架构中,通过在区块结构上的良好设计,基础协议将会成为链上数据的一部分,并通过链上数据的追加实现基础协议的升级;对于星云链中的应用(智能合约),星云链通过在智能合约底层存储支持状态变量可跨合约
    访问的设计,完成智能合约的升级。具备自我升级进化能力的星云链,未来比其它的公有链具有更快的发展速度和更大的生存潜力,同时使得开发者面对漏洞,能够更快的响应 和升级,避免黑客事件给用户带来巨大的损失。

\item 区块链应用生态环境的建设

在星云链中,我们提出了基于账户贡献度的PoD(见\refsec{sec:pod})算法,利用NR的价值尺度评估找出对生态贡献度较高的账户,平等地赋予记账资格,遏制记账权被垄断,并且融合PoS中的经济惩罚,防止公链被恶意破坏,为生态自由发展助力。既能保证较快的共识速度,又能比PoS和PoI更抗作弊,对区块链生态的发展有良好的促进作用。


在星云链中,我们提出面向智能合约和DApp开发者的DIP(Developer Incentive Protocol, 开发者激励协议)(见\refsec{sec:dip})。DIP的核心思想是对社区贡献度高的智能合约或DApp的开发者,给予他们相应的开发者激励。激励由记账人负责写入区块。

基于Nebulas
Rank机制,星云链进一步包含了搜索引擎(见\refsec{sec:search}),以帮助用户更好的探索区块链中的高价值应用。

\end{itemize}


考虑到以太坊已经有巨大的生态,是一个非常成功的公有区块链平台。星云链希望尽可能的借鉴以太坊优秀的设计,从协议底层以及智能合约编程上完全兼容以太坊,使得基于以太坊开发的产品能够零成本的迁移到星云链上。


基于上述设计原则,我们试图构建一个基于价值激励的去中心化应用平台。本白皮书详细描述了星云链中关于技术的细节,其中\refsec{sec:rank}描述了一种可能的价值尺度及其算法Nebulas
Rank,\refsec{sec:nebulasforce}描述了星云链的自我进化能力Nebulas Force,
\refsec{sec:dip}、\refsec{sec:pod}、\refsec{sec:search}、\refsec{sec:tools}描述了星云链在建设区块链应用生态圈的的努力与尝试,最后,\refsec{sec:nascoin}描述了星云链的代币NAS。
