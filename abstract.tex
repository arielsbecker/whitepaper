\abstract{
比特币和以太坊系统分别给区块链世界带来“去中心化现金”和“智能合约”技术。相对于3年前,区块链技术及产业已经取得了长足的发展和繁荣。但是我们发现,在和现有商业及用户需求逐步结合的同时,已有的技术并不完美,面对着越来越多的挑战。

本文介绍了星云链的技术架构设计,意图构建一个基于价值尺度,并带有价值激励机制,具备自进化能力的区块链系统,主要内容包括:

\begin{itemize}
	\item \textbf{定义价值尺度的Nebulas Rank}(\refsec{sec:rank}),通过综合考虑链中各个账户的流动性、传播性和互操作性,Nebulas Rank试图为每个账户建立一个可信、可计算和可复现的普适价值尺度刻画。利用NR,将会有更多的应用可以结合自身场景,来挖掘更大纵深的公链价值。
	\item \textbf{基于价值尺度的PoD贡献度证明共识算法}(\refsec{sec:pod}),为了公链生态健康自由发展出发,星云链提出了共识算法的三个重要指标,即快速、不可逆和公平性,PoD通过融合PoS和PoI的优势,结合NR,在优先保证快速和不可逆的前提下,率先加入了公平性的考量。
	\item \textbf{支持核心协议和智能合约链上升级的星云原力Nebulas Force}(\refsec{sec:nebulasforce}),帮助公链本身和链上应用实现自我进化,动态调优适应市场变化,将会有更快的发展速度和更大的生存潜力,有能力构建更宏伟的愿景。
	\item \textbf{DIP开发者激励协议}(\refsec{sec:dip}),为了更好地建立区块链应用生态环境,星云链将通过星云币的奖励来感谢为⽣态助⼒的优秀智能合约开发者,促进公链更加丰富多元的价值沉淀。
	\item \textbf{智能合约搜索引擎}(\refsec{sec:search})。
\end{itemize}
}