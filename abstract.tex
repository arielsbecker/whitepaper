\abstract{
  比特币和以太坊系统分别给区块链世界带来“去中心化现金”和“智能合约”技术。如今,区块链技术及产业已经取得了长足的发展和繁荣,各种应用场景、商业需求层出不穷。随之而来的,我们发现,已有的区块链技术已经不能满足日益增长的用户需求,总的来说,区块链技术面临着价值尺度缺失、自我进化及生态建设三方面的挑战。

  本文介绍了星云链的技术架构设计,意图构建一个能够量化价值尺度、具备自进化能力,并能促进区块链生态建设的区块链系统,主要内容包括:

\begin{itemize}
  \item \textbf{定义价值尺度的Nebulas
      Rank(NR)}(\refsec{sec:rank}),通过综合考虑链中各个账户的流动性及传播性,NR试图为每个账户建立一个可信、可计算及可复现的普适价值尺度刻画。可以预见,在NR之上,通过挖掘更大纵深的价值,星云链的平台上将会涌现更多、更丰富的应用。

	\item \textbf{支持核心协议和智能合约链上升级的星云原力Nebulas
      Force}(\refsec{sec:nebulasforce}),帮助星云链本身及其上的应用实现自我进化,动态适应市场变化,从而使得星云链及应用将会有更快的发展速度和更大的生存潜力,开发者亦能够通过星云链构建更丰富的应用,并进行快速迭代。
	\item
    \textbf{DIP开发者激励协议}(\refsec{sec:dip}),为了更好地建立区块链应用生态环境,星云链将通过星云币来激励为⽣态助⼒的优秀应用开发者,促进星云链更加丰富多元的价值沉淀。
  \item \textbf{贡献度证明(Proof of Devotion,
      PoD)共识算法}(\refsec{sec:pod}),从星云链生态健康自由发展出发,星云链提出了共识算法的三个重要指标,即快速、不可逆和公平性,PoD通过融合PoS和PoI的优势,结合星云链中的价值尺度,在保证快速和不可逆的前提下,率先加入了公平性的考量。
  \item
    \textbf{去中心化应用的搜索引擎}(\refsec{sec:search}),基于我们所定义的价值尺度,星云链构建了一个针对去中心化应用的搜索引擎,帮助用户在海量区块链应用中,找到符合用户期望及应用场景的应用。

\end{itemize}
}
