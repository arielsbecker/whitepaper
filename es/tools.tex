\section{Fundamental Services and Development Tools}
\label{sec:tools}

\subsection{Domain Name Service}

Due to the anonymity of blockchains, account addresses are long and meaningless strings, which are not user-friendly. For this reason, users are prone to misoperations like money loss caused by unintentional funding or the interaction with incorrect objects. In other words, by using domain names that are easy to remember, users can gain better experience. By using smart contracts, the Nebulas development team will implement a DNS-like domain system named Nebulas Name Service (NNS) on the chain while ensuring that it is unrestricted, free and open. Any third-party developers can implement their own domain name resolution services independently or based on NNS.

For example, Alice applied for the domain name ``alice.nns" for her account address \\0xdf4d22611412132d3e9bd322f82e2940674ec1bc03b20e40. To transfer money to Alice, Bob simply needs to enter ``alice.nns" in payee information so that the money will be transferred to the correct payee address through the NNS service.


NNS application rules are as follows:

\begin{itemize}

	\item Top-level sub-domain names will be reserved and unavailable for application, such as *.nns, *.com, *.org and *.edu. Thus, users can only apply for second-level sub-domain names.
	\item Once the NNS service is active, users can query domain names for availability. For a vacant domain name, users can bid for it through a smart contract. The bidding process is open so that any user can query for others' bids and update their own bids anytime.
	\item Once the bidding period expires, the highest bidder wins the domain name and the smart contract locks the user's bidding funds. The validity period of the domain name is one year. One year later, the user can freely determine whether to renew or not. If yes, the validity period will be extended for another year. If no, the bidding funds will be refunded to the user's account and the domain name will be released as available again.
	\item Users can give up the ownership of a domain name at any time. In this case, the bidding funds will be automatically refunded to the user's account and the domain name will be released as available again after domain data is cleared.
	\item Users can transfer the ownership of a domain name with or without compensation. Nebulas does not intervene in any transactions of domain names.

\end{itemize}


\subsection{Lightning Network}

Currently, all public blockchain networks are faced with system scalability challenges. For example, Bitcoin network is only able to process 7 transactions per second and Ethereum is able to process 15 transactions per second. By introducing PoS-like consensus algorithms, mining calculation can be avoided and the consensus speed can be improved dramatically. However, public blockchains are still greatly challenged by massive micro-payment scenarios in the real world. Put forward in February 2015, the lightning network \cite{poon2015bitcoin} was designed to set up a channel network for micro-payment between transaction parties, so that large amounts of payments between the parties can be confirmed off the blockchain directly, repeatedly, frequently and bi-directionally in the netting method. When a transaction needs to be settled, the final result will be submitted to the blockchains for confirmation. Theoretically, this can achieve millions of transfers per second. If no point-to-point payment channel is available between the parties, a payment path connecting both parties and consisting of multiple payment channels can also be used to achieve reliable fund transfer between the parties. The lightning network has gone through the PoC phase on both Bitcoin and Ethereum.

Nebulas implements the lightning network as the infrastructure of blockchains and offers flexible design. Any third-party developers can use the basic service of lightning network to develop applications for frequent transaction scenarios on Nebulas. In addition, Nebulas will launch the world's first wallet app that supports the lightning network.

\subsection{Developer Tools}

A complete set of developer tools are critical to blockchain app developers. So far, developer tool chains for pubic blockchains are incomplete, imposing great challenges to most developers. Nebulas development team will provide a rich set of developer tools, including independent development IDE for smart contracts, block browser, plugins for various popular IDEs (including Eclipse, JetBrains, Visual Studio, Sublime Text, VIM and Atom), debugger, simulator, formal verification tool for smart contracts, background SDKs for various advanced languages, and SDKs for mobile ends.