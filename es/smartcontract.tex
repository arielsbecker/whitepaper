\subsection{Actualización del diseño de los contratos inteligentes}

\subsubsection{Lenguaje de programación Turing-completo de contratos inteligentes}

Un contrato inteligente es un conjunto de \textit{promesas} definidas de forma digital, lo que incluye acuerdos de contrato para la ejecución de esas promesas por parte de los participantes. Físicamente, el soporte del contrato inteligente es un código informático que una computadora puede reconocer y operar. El lenguaje de scripting de Bitcoin es imperativo y basado en pila, por lo que es Turing-incompleto y sus aplicaciones son limitadas. Ethereum es el primer caso en el mundo de un sistema blockchain que implementa contratos inteligentes Turing-completos; adopta Solidity y Serpent como lenguajes, lo que les permite a sus desarrolladores escribir una amplia variedad de aplicaciones de forma rápida y sencilla. Una vez que el código de un contrato inteligente se publica es posible ejecutarlo automáticamente sin intervención de ningún organismo.

En sus primeras etapas, el lenguaje de programación de contratos inteligentes de Nebulas era totalmente compatible con Solidity, lo que facilitaba a los desarrolladores la migración de sus aplicaciones de Ethereum a Nebulas. Añadimos a nuestra implementación de Solidity un conjunto de instrucciones específicas relacionadas a Nebulas Rank con el fin de facilitarles a los desarrolladores la obtención de la valuación NR de cualquier usuario. Luego, basándonos en nuestra NVM, comenzamos a brindar soporte a varios lenguajes de programación, como Java, Python, Go, JavaScript, Scala, etc., con el fin de que cualquier desarrollador familiarizado con ellos tenga la posibilidad de escribir su propio contrato inteligente o DApp.

\subsubsection{Contratos actualizables}

Actualmente, al diseñar un contrato inteligente en Ethereum, no es posible cambiar el código una vez que se ha implementado en la red. En un principio los contratos inteligentes servían como un acuerdo propiamente dicho, por lo cual era razonable entender el diseño como inmutable una vez implementado; el problema está en que, si los desarrolladores no prueban debidamente el código, una vez implementado, será imposible actualizar su código.

No obstante, a medida que los contratos inteligentes se popularizaron, su flujo de trabajo y su código se hicieron cada vez más complicados. En cuanto un hacker encuentra un error, es muy difícil prevenir las pérdidas. En junio de 2016 ocurrió el famoso ataque DAO, producto de un error en el código, lo que causó una pérdida total de \$60 millones de dólares entre los usuarios de Ethereum. A eso debe sumársele un error en la cartera \textit{Parity} que también resultó en la pérdida de 150 000 ETH, valuados en ese momento en \$30 millones de dólares. Debido a que Bitcoin fue diseñado como un sistema Turing-incompleto y muchas de las instrucciones de scripting no están disponibles, su nivel de seguridad es muy alto.

Aunque existen muchas buenas prácticas en la programación de contratos inteligentes, procesos de revisión estrictos e incluso herramientas de verificación formal para asegurar el buen funcionamiento del contrato inteligente a través de pruebas matemáticas, nunca será posible afirmar que un código está totalmente libres de errores. Cuando observamos la historia de Internet, encontramos que para todos los errores graves siempre hubo una actualización disponible en breve. Por ello, creemos que el requisito fundamental para resolver el problema de seguridad de los contratos inteligentes es formular para ellos una solución de diseño actualizable.

Existen dos soluciones propuestas al problema de la inmutabilidad de los contratos inteligentes en Ethereum: la primera es permitir que el Proxy Contract esté disponible al público. Su código es muy sencillo: sólo se encarga de reenviar las peticiones al verdadero contrato inteligente. Cuando se requiere una actualización del contrato inteligente, sólo es necesario hacer que el puntero del \textit{Proxy Contract} apunte al nuevo contrato. La segunda es hacer que el código y el almacenamiento del contrato queden separados. El almacenamiento es responsable de proveer al contrato externo de métodos de lectura y escritura para el estado interno; el código del contrato es responsable de la lógica real, y al llevarse a cabo una actualización, sólo es responsable de implementar el nuevo código del contrato, de modo de no perder el estado actual. Las dos soluciones tienen sus propias limitaciones, por lo que no pueden resolver todos los problemas. Por ejemplo, la separación del código y el almacenamiento hace que el desarrollo de contratos inteligentes sea más complejo, tanto que en ocasiones resulta imposible. Aún cuando el \textit{Proxy Contract} es capaz de apuntar a los nuevos contratos, los datos de estado del contrato anterior no se pueden migrar; a su vez, algunos contratos no fueron bien diseñados en la primera etapa de desarrollo, por lo que no podrán adaptarse para este mecanismo de actualizaciones.

En nuestro caso, diseñamos una solución sencilla al problema de la actualización de contratos que consiste en que, a nivel de lenguaje, cualquier otro contrato con las credenciales adecuadas puede leer y escribir directamente en las variables de estado. Por ejemplo, para un contrato de \textit{tokens} cuyo código se muestra en la figura \ref{figure:nf:oldsc}. \\

	\begin{figure}[!h]
  	\centering
  	\begin{minipage}{0.95\linewidth}
	\begin{lstlisting}[frame=single]
contract Token {
  mapping (address => uint256) balances shared;

  function transfer(address _to, uint256 _value) returns (bool success) {
     if (balances[msg.sender] >= _value) {
       balances[msg.sender] -= _value;
       balances[_to] += _value;
       return true;
     } else {
       return false;
     }
   }
   function balanceOf(address _owner) constant returns (uint256 balance) {
       return balances[_owner];
   }
}
	\end{lstlisting}
  	\end{minipage}
  	\caption{Código de contrato original}
  	\label{figure:nf:oldsc}
	\end{figure}

Al implementar el contrato, las variables de tipo balance se marcan con la palabra clave \texttt{shared}, y al compilarlo como bytecode, la máquina virtual crea un área de almacenamiento separada para este tipo de variables. Las variables que no están marcadas como \texttt{shared} no pueden ser leídas directamente por otros contratos.

Si existe un error en la función de transferencia del código original y es necesario modificarlo, se debe chequear la variable \_value e implementar el nuevo código del contrato inteligente tal como se muestra en la figura \ref{figure:nf:newsc}.

	\begin{figure}[!h]
  	\centering
  	\begin{minipage}{0.95\linewidth}
	\begin{lstlisting}[frame=single]
[baseContractAddress="0x5d65d971895edc438f465c17db6992698a52318d"]
//baseContractAddress is the address of the old contract
contract Token {
  mapping (address => uint256) balances shared;

  function transfer(address _to, uint256 _value) returns (bool success) {
     if (balances[msg.sender] >= _value ^&& _value > 0^) {
       balances[msg.sender] -= _value;
       balances[_to] += _value;
       return true;
     } else {
       return false;
     }
   }
   function balanceOf(address _owner) constant returns (uint256 balance) {
       return balances[_owner];
   }
}
	\end{lstlisting}
  	\end{minipage}
  	\caption{Nuevo código de contrato}
  	\label{figure:nf:newsc}
	\end{figure}

Luego de implementar el nuevo contrato, el anterior se puede autodestruir —lo que significa que ya no será posible acceder al mismo—, pero todas las variables \textit{shared} se mantendrán intactas. El nuevo contrato puede heredar por completo los balances del viejo contrato sin perder ningún estado, por lo que no se requiere ningún paso adicional en la migración. No obstante ello, al desarrollar un contrato inteligente, es necesario declarar explícitamente las variables críticas de estado como \textit{shared}. El compilador manejará especialmente el área de almacenamiento de esas variables de modo de asegurar que sean accesibles a cualquier otro contrato autorizado.

Para garantizar la seguridad, el nuevo contrato debe provenir del mismo creador que el contrato original; de no ocurrir así, surgirá una excepción durante la operación, que no podrá completarse.

Existe un problema moral con este sistema: una vez que un contrato se ha propuesto, se han tomado en recaudo todas las previsiones, se ha escrito y se ha finalizado, el mismo no debería sufrir más modificaciones, y en caso de requerirse alguna corrección técnica (por un error en el código, por ejemplo), esa modificación debería someterse a la decisión de una audiencia. Por ello, planeamos implementar un mecanismo de voto para aprobar la actualización de todo contrato inteligente, impidiendo así que sus autores realicen modificaciones sobre el mismo sin dar aviso a sus usuarios.

Con esta solución para las actualizaciones, el ataque DAO o el error de paridad —o cualquier ataque o error similar— podrán ser corregidos de forma mucho más rápida, sin necesidad de recurrir a ningún \textit{fork}. Luego de aplicar la actualización, los activos de todos los usuarios podrán seguir utilizándose sin necesidad de realizar ninguna migración.