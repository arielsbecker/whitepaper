\section{Conclusión}
\label{sec:conclusion}

\subsubsection*{Nuestra postura}

Desde una perspectiva altamente abstracta, un blockchain es \textbf{una forma descentralizada de determinar los datos}, y sus tokens funcionan como \textbf{el soporte para esa determinación}. Mientras Internet resuelve la necesidad de la transmisión de datos, Blockchain va más allá, y resuelve la determinación de esos datos. Por primera vez en la historia, Blockchain convierte los datos públicos en datos privados que ya no serán analizados y utilizados arbitrariamente por grandes empresas como Google, Amazon y Facebook.

La esencia de este nuevo paradigma —representada por los blockchains públicos— es la conjunción de tres elementos: \textbf{comunidad, token, y aplicación}. La \textbf{comunidad} es, esencialmente, un ecosistema construido de abajo hacia arriba, que adhiere a la idea de apertura, código abierto, intercambio y no lucratividad, que es completamente diferente de los ecosistemas empresariales construidos de arriba hacia abajo. El \textbf{token} es el portador del valor de los datos; en el futuro habrá más escenarios de uso para los blockchains que los de moneda virtual y efectivo electrónico. La \textbf{aplicación} se refiere simplemente a la implementación tecnológica de los escenarios de aplicación de los blockchains. Sin la combinación de los dos primeros factores mencionados, una aplicación por sí sola no puede reflejar completamente el atractivo de los sistemas blockchain.

Los atributos de los blockchains públicos (\textbf{veracidad y transparencia}) representan el valor real de estos sistemas. Por el contrario, la mayoría de los blockchains de empresas y consorcios son permisionados o privados, lo que significa que no pueden romper los patrones existentes, y son consideradas como meras innovaciones mejoradas, mientras que los blockchains públicos anulan las relaciones de confianza existentes y se consideran como \textbf{innovaciones revolucionarias}, que reflejan el valor máximo de los blockchains.

\subsubsection*{Nuestro compromiso}

Al ser el primer motor de búsqueda en blockchains del mundo, Nebulas está comprometido a \textbf{explorar las dimensiones ocultas del valor} y en crear sistemas operativos en el blockchain basados en valor, motores de búsqueda y otras aplicaciones relacionadas.

Con este compromiso presentamos \textbf{Nebulas Rank}, un estándar de valuación del mundo blockchain; diseñamos \textbf{Nebulas Force} para implementar la capacidad auto-evolutiva en los blockchains; desarrollamos los sistemas \textbf{Developer Incentive Protocol} y \textbf{Proof of Devotion} para motivar la actualización del valor de los blockchains; y construimos \textbf{Nebulas Search Engine} para ayudar a nuestros usuarios a explorar otras dimensiones en el valor de los blockchains.

\subsubsection*{Nuestras creencias}

La continua evolución científica y tecnológica nos llevará a una vida mejor con un mayor nivel de \textbf{libertad e igualdad}. Como una de las principales tecnologías de nuestro tiempo, los blockchains gradualmente nos mostrarán todo el brillo de sus ventajas. Ser parte de esta evolución es nuestra mayor felicidad y logro.

Al igual que en Internet, los blockchains también entrarán en una fase explosiva en cuanto a cantidad de usuarios y aplicaciones. La tecnología del blockchain se convertirá en el \textbf{protocolo central} de la próxima generación de la red inteligente, y el número de usuarios alcanzará o incluso superará los mil millones en los próximos 5 a 10 años. En los próximos cinco años surgirán importantes oportunidades y desafíos.

Frente al vasto ecosistema del futuro, no nos debemos preguntar qué puede hacer blockchain por nosotros, sino qué podemos hacer nosotros por blockchain, porque \textbf{los blockchains son a la vez un organismo y una economía}.

Estamos encantados de compartir con todos ustedes la exploración de las tecnologías blockchain.