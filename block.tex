\subsection{核心协议的升级设计}
交易是区块链中最重要的组成部分,用户和区块链上的互动都是通过特定的交易进行的。用户创建一笔交易,用自己的私钥签名后,发到区块链的任何一个节点中,通过P2P网络广播到全网节点。在固定的出块时间间隔内,由PoD共识算法(见\refsec{sec:pod})指定的记账节点累计这段时间内的所有交易,打包成标准格式的区块,同步到全网所有节点。经各节点独立验证后,加入本地账本,成为全球总账本的一部分。

在以太坊中,交易分为两种类型:普通用户转账交易,智能合约交易。我们在星云链的系统架构中,增加特殊的第三种交易类型:核心协议代码。核心协议代码作为区块链数据的一部分存储在链上。星云链基础协议的升级,是通过链上数据的追加而实现的。

星云链区块数据结构包含,但不限于以下信息:
\begin{itemize}
	\item Header:区块头
		\begin{itemize}
		\item Height:高度
		\item ParentHash:父区块哈希值
		\item Ts:时间戳
		\item Miner:记账人地址
		\item StateRoot:状态根哈希值
		\item TxsRoot:交易根哈希值
		\item ReceiptsRoot:交易收据根哈希值
		\item TransNum:交易数
		\end{itemize}
	\item Transactions:交易数据(包含多个交易)
		\begin{itemize}
		\item From:交易发起人地址
		\item To:交易接收人(普通用户或智能合约)地址,对于创建智能合约,值为0地址
		\item Value:转账金额
		\item Data:交易的payload。如果交易为创建智能合约,则为智能合约字节码;如果交易为智能合约调用,包含调用函数名称和入参值;其余可以为任何值。
		\item Signature:交易签名
		\item Gas:燃料上限
		\item GasPrice:燃料单价
		\item Nonce:标识交易唯一性
		\end{itemize}
	\item Protocol Code:核心协议代码(一个区块中只能有0或1个)
		\begin{itemize}
		\item Hash:哈希值
		\item Code:核心协议的字节码,目前考虑为JVM字节码。
		\item Signature:签名(校验是否来自开发者社区保留账号的签名)
		\item Version:标识核心协议版本号,每次升级都需要递增,防止恶意记账人回滚到老的Protocol Code。
		\item Nonce:标识唯一性
		\end{itemize}
\end{itemize}

星云链客户端节点从当前最新区块的Protocol Code存储区可以取到编译后的虚拟机字节码(目前考虑为JVM),如果当前最新区块没有Protocol Code数据,说明核心协议没有变更,就往前追溯到最近区块的Protocol Code。区块链的核心协议行为都由Protocol Code确定,包括验证算法、打包规则、NR算法、奖励机制等,几乎绝大部分的区块链行为都可以由Protocol Code定义。在创世区块中已经包含第一个版本的完整的核心协议Code。后续如果核心协议需要升级,由星云链团队开发,把代码在公开渠道让社区讨论和投票。投票可以通过智能合约或者论坛投票的形式进行,当绝大部分社区成员都同意协议升级,星云链开发组把最新代码打包成Protocol Code交易,发布到全网节点,记账节点只要把其包含进区块,就可以在下一个区块生效。这种方式的区块链协议升级,对客户端来说是透明的,无需软、硬分叉。

为了保证核心协议代码是经过授权发布的,Protocol Code的发布者是星云链核心开发组保留地址,该地址在创世区块内部硬编码无法变更。所有记账节点都会验证Protocol Code签名,签名不通过的视为非法数据。后续可以把Protocol Code的签名校验改成M-of-N的多签名形式,这个本身也可以通过Protocol Code的升级实现。