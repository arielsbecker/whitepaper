\section{总结}
\label{sec:conclusion}

\subsubsection*{我们认为}

区块链从一种高度抽象的角度来看是一种\textbf{用去中心化的方式对于数据的确权},代币本身是对于\textbf{确权价值的载体}。互联网解决了数据的通讯问题,而区块链则在互联网上层进一步解决了数据的确权问题。区块链前所未有第一次让大家的数据真正变成自己的数据,而不再被BAT等大公司任意分析并使用。

以公有链为代表的区块链本质精神是:\textbf{社群+代币+工具}。社群本质上是自下而上,秉承的是开放、开源、共享、非盈利的理念,和现有自上而下的商业生态有着根本的不同。代币即是对于确权价值的载体,未来会有更多的应用场景,而远非是仅仅面向虚拟货币,电子现金的属性。工具仅仅是对于区块链应用场景具体技术实施,如缺少前两者的结合,其单独来看\textbf{并不能完全体现区块链系统的魅力所在}。

以公有链为代表的区块链体系才是区块链的未来,因为其本身所具有的“\textbf{非信任}”、“\textbf{无特权}”的基本特性才是区块链系统真正的价值所在。恰恰相反,作为联盟链/企业链大多具有“\textbf{基于信任}”和“\textbf{基于特权}”的属性,不能突破既有的范式,属于改良式创新。而公有链系统颠覆了既有的协作关系,属于\textbf{颠覆式创新},是区块链价值最大化的真正体现。

\subsubsection*{我们致力于}

作为全球首个区块链搜索引擎,星云链致力于\textbf{发掘区块链世界价值新维度},打造基于价值尺度的区块链操作系统、搜索引擎及其他相关扩展应用。

基于此,我们提出Nebulas Rank星云指数来构建区块链世界的价值尺度,设计Nebulas Force星云原力来赋予区块链自我进化的能力,推出Developer Incentive Protocol开发者激励计划和Proof of Devotion贡献度共识证明来激励区块链的价值升级,打造Nebulas Search Engine区块链搜索引擎来帮助用户发现区块链上沉淀的多维度价值。

\subsubsection*{我们坚信}

正在发生科技浪潮会带领人们抵达更为\textbf{自由、平等、和美好}的生活。区块链作为其中重要的技术之一,会愈来愈散发出其特有的光彩和能量。能参与并投身其中,是我们最大的快乐和成就。

类似互联网对于世界的改变,区块链也即将面对其用户/应用临界值爆发的阶段。区块链技术会是下一代“智能网络”的\textbf{基础协议},整体用户规模会在5-10年内达到或超过10亿。未来的5年内会面临重大的机遇与挑战!

在未来的巨大生态面前,在当下,不要问区块链能为你做什么,而是要问你能为区块链做什么。因为,\textbf{区块链本身就是生命体,区块链本身就是经济体}!在区块链技术探索的道路上,与诸君共勉!
