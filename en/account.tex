\section{Nebulas Account Address Design}

We carefully design Nebulas account address to make it start with ``n'',
which
is short for ``Nebulas''.

\subsection{Account Address}

Similar to Bitcoin and Ethereum, Nebulas also adopts elliptic curve algorithm as its basic encryption algorithm for Nebulas accounts.
The address is derived from public key, which is in turn derived from the private key that encrypted with user's passphrase.

Also we have the checksum design aiming to prevent a user from sending NAS to a wrong user account accidentally due to entry of several incorrect characters.
The specific calculation method is provided as follows:

\begin{verbatim}
content = ripemd160(sha3_256(public key))
checksum = sha3_256(0x19<<(21*8) + 0x57<<(20*8) + content)[0:4]
address = base58(0x19<<(25*8) + 0x57<<(24*8) + content<<(4*8) + checksum)
\end{verbatim}
\noindent in which 0x19 is for padding, and 0x57 specifies the type of
address. Notice that the length of \texttt{content} is 20 bytes, the length of
\texttt{checksum} is 4 bytes, and the length of \texttt{address} is 26 bytes.

A Nebulas address contains a total of 35 characters including the prefix
``n'', which is encoded with base58. A typical address looks like this
\texttt{n1TV3sU6jyzR4rJ1D7jCAmtVGSntJagXZHC}.


\subsection{Smart Contract Address}
Calculating contract address differs slightly from account, passphrase of
contract sender is not required but address \& nonce.  Calculation formula is as follows:

\begin{verbatim}
content = ripemd160(sha3_256(tx.from, tx.nonce))
checksum = sha3_256(0x19<<(21*8) + 0x58<<(20*8) + content)[0:4]
address = base58(0x19<<(25*8) + 0x58<<(24*8) + content<<(4*8) + checksum)
\end{verbatim}
\noindent
in which 0x58 indicates the address type for smart contract. A typical smart
contract address is like \texttt{n1sLnoc7j57YfzAVP8tJ3yK5a2i56QrTDdK}.
