\section{Introduction}

\subsection{Blockchain technology introduction}
%\subsection{区块链技术简介}
Blockchain technology is derived from a decentralized digital currency, Bitcoin, which was conceptualized by Satoshi Nakamoto in 2008~\cite{Nakamoto2008}. Instead of being issued by any institutions, Bitcoin is generated through specific algorithm and massive computing to ensure the consistency of the distributed ledger system. Ethereum~\cite{buterin2014ethereum} goes further and provides a public blockchain-based computing platform with a Turing-complete language. The core of cryptocurrency systems represented by Bitcoin and Ethereum is blockchain technology. With data encryption, timestamp, distributed consensus and economic incentive, blockchain technology brings into reality peer-to-peer transaction, coordination and collaboration in the distributed system in which the nodes do not need to trust each other, resolving common issues faced by centralized institutions including high cost, low efficiency and insecure data storage.
%2008年10⽉31⽇,中本聪(Satoshi Nakamoto)提出了⽐特币~\cite{Nakamoto2008}的设计⽩⽪书,从此我们迎来了⼀个有区块链的世界。⽐特币作为区块链的太初应⽤,践⾏了其作为"⼀个去中⼼化电⼦现⾦系统"的初衷。比特币的产生不依赖于任何机构,而是根据特定算法,依靠大量计算产生,保证了比特币网络分布式记账系统的一致性。以太坊Ethereum~\cite{buterin2014ethereum}更进一步,为我们提供了一个可以运行具有图灵完备性的代码的通用区块链框架。区块链是以比特币和以太坊为代表的数字加密货币体系的核心支撑技术,通过运用数据加密、时间戳、分布式共识和经济激励等手段,在节点无需互相信任的分布式系统中实现去中心化信用的点对点交易、协调与协作,从而解决中心化机构普遍存在的高成本、低效率和数据存储不安全等问题。

It should be noted that the blockchain technology itself is not a brand new technological innovation, but a model innovation that combines a series of technologies including peer-to-peer communication, cryptography, block-chain data structure, etc.
%需要说明的是,区块链技术本身不是一个全新的技术创新,而是作为一系列技术的组合(包括点对点通讯,密码学,块链数据结构等)产生的模式创新。

\subsection{Business and technology challenges}
%\subsection{商业和技术挑战}
As more people join in the development camp of decentralized self-governing systems represented by blockchain technology, the world has seen a dramatic increase of blockchain projects to over 2,000 with the value of global encrypted digital assets amounting to \$90 billion dollars. The number of blockchain users/digital asset owners is also rising rapidly from 2 million in the beginning of 2013 to 20 million in early 2017. By 2020, the number of blockchain users/digital asset owners is expected to reach or surpass 200 million, and by 2025, 1 billion.
%以区块链技术为代表的去中心化,自主治理的系统,正在引起越来越多人的重视和研究。当前全球区块链项目已经超过2000个,全球加密数字资产总体价值达到900亿美元。区块链/数字资产领域的用户人群也正在快速增加。从2013年初的全球200万用户,到2017年初的2000万用户。我们认为,在2020年左右,全球区块链/数字资产用户会达到或超过2亿。在2025年前后,全球用户有望达到10亿规模。

With the popularity of blockchain technology, more blockchain-based application scenarios are emerging. Such scenarios have gradually extended from digital currency to a broader range and user-group coverage. For example, Ethereum community introduced the concept of smart contract to blockchain technology; Ripple implemented a global settlement system using blockchain technology. Diverse application scenarios also come with increasing demands and challenges.
%随着区块链技术的普及,越来越多区块链技术之上的应用场景被挖掘出来。区块链技术的应用场景已经从最初的数字化货币本身逐步扩展到更多的场景及用户群体中。例如,以以太坊为代表的社区在区块链技术中引入智能合约的概念,Ripple则使用区块链技术实现了全球的结算系统。随着应用场景的多样化,用户对区块链技术的诉求也日益增加,我们已经看到很多挑战。

\paragraph{Lack of measure of value.} We believe that the blockchain world needs a universal measure of value to measure the value of users and smart contracts. Upper-layer applications can be built on this universal measure of value to seek deeper value in its particular scenario. In this sense, innovations in business models will abound in the future, reminiscent of Google’s rise in the world of Internet.
%\paragraph{价值尺度的缺失}我们认为,区块链世界需要一个普适的价值尺度,来衡量用户和智能合约的价值,上层应用可以在这个普适的价值尺度上结合自身场景挖掘更深层次的价值,这将带来更多的商业模式的创新,就像Google在互联网世界的兴起一样。

\paragraph{Blockchain system upgrade.} Unlike common software, the decentralized blockchain system cannot enforce users to upgrade clients and protocols. Therefore, protocol upgrade in the blockchain system often leads to 	``hard fork" or ``soft fork" and results in huge losses, which further limits the application of the system. In the case of Bitcoin, controversy still abounds over block scaling within the community, which hinders the evolution of Bitcoin protocols. The severe undercapacity of block has once led to a situation where nearly one million transactions were waiting in the transaction pool to be written into the blocks. Users often have to pay an extra high ``transaction acceleration fee", which seriously affect user experience. Moreover, although Ethereum's ``hard fork" offers a temporary fix to The DAO problem, it also gives rise to ``side effects" like ETH and ETC ``duplicate assets" and division of community.
%\paragraph{区块链系统的升级}不同于普通软件的版本迭代,区块链系统由于其天生的去中心化特性,无法强制用户升级其客户端及协议。因此,区块链系统中的协议升级往往会引发区块链"硬分叉"或"软分叉",从而造成巨大的损失,这更进一步限制了区块链系统的应用场景。以比特币为例,社区关于区块扩容至今仍然存在巨大的争议,导致比特币协议进化缓慢,区块容量严重不足,出现过近100万笔交易在交易池等待被写入区块。用户很多时候不得不额外支付高昂的"交易加速费",严重损害体验性能。另外,从以太坊的"硬分叉"来看,虽然暂时解决了The DAO问题,但是产生了ETH和ETC"重资产"和社区分裂的"副作用"。

\paragraph{The construction of blockchain application ecosystem.} With applications (DApp) increasing rapidly on the blockchain, a sound ecosystem becomes the key to better user experience. What we should think about is how to help users search and find the desired DApp in a massive collection of blockchain applications, how to encourage developers to develop more DApps for users, and how to assist developers with the construction of DApps. Take Ethereum for example. Hundreds of thousands of apps have been built on Ethereum now, and when the number of blockchain DApps reaches that of Apple App Store, how to search and find the expected DApp would be a big challenge.
%\paragraph{区块链应用生态环境的建立}随着区块链上各种应用(DApp)的快速增长,良好的生态环境是提高用户体验的根本所在。这包括用户如何在海量的区块链应用中检索自己期望的DApp,如何激励开发人员为用户提供更多的DApp,以及如何帮助开发人员更快的构建更好的DApp。以以太坊为例,基于以太坊的DApp总数已经数十万个,试想如果区块链世界中的DApp接近苹果App Store里应用总量规模的话,如何发现并找到用户期望的DApp就是个很大问题。


\subsection{Nebulas design principles}
%\subsection{星云链设计原则}
We set out to design an incentive-based and self-evolving blockchain system to take up those challenges and opportunities. The design principles are as follows:
%面对上述机遇和挑战,我们要设计一个基于价值激励的自进化区块链系统。具体来说,我们有以下设计原则:
\begin{itemize}
	\item \textbf{A fair ranking algorithm to define the measure of value}
	%\item \textbf{公正的排名算法,定义价值尺度}

We believe that the blockchain world needs a universal measure of value to measure the value of simple data at the bottom layer of blockchain to identify a higher dimension of information, thus exploring greater value of the blockchain world. We put forward the NR (Nebulas Rank) ~(see \refsec{sec:rank}) algorithm (similar to Google's PageRank~\cite{Brin2010}\cite{page1999pagerank}), which combines the liquidity, speed, width and depth of capital on the blockchain to provide a fair ranking for blockchain users. NR is the measure of value in the blockchain world, with which developers can measure the importance of each user, smart contract, and DApp in different scenarios. NR has huge commercial potential and can be used in search, recommendation, advertising and other fields.
%我们认为,区块链世界需要一个普适的价值尺度,用于衡量区块链底层简单数据的价值,发现信息的更高维度信息,从而探索并挖掘区块链世界的更大价值。类似Google的PageRank~\cite{Brin2010}\cite{page1999pagerank},我们也提出区块链世界的NR(Nebulas Rank,星云指数)~(见\refsec{sec:rank})算法,综合考虑了区块链上的资金流动性,以及资金传播的速度、广度和深度,给区块链用户做公正的排名。NR是星云链赋予区块链世界的价值尺度,用来帮助开发者结合自身场景有效衡量区块链中各个用户、智能合约、DApp的重要性。NR有巨大的商业潜力,可以用在搜索、推荐、广告等领域当中。

\item \textbf{The self-evolving of blockchain system and applications}
%  \item \textbf{区块链系统及应用的自我进化}

We believe that a well-conditioned system and the applications on it should be able to self-evolve, which means to achieve faster computing, better resilience, and enhanced user experience under little intervention. We call this self-evolving ability ``Nebulas Force" ~(see \refsec{sec:nebulasforce}). In Nebulas’ system architecture, thanks to our well-designed block structure, base protocols will become part of the data on the blockchain and achieve upgrade through the addition of data. As for applications (smart contracts) on Nebulas, we make the upgrade of smart contracts possible by enabling cross-contract access to state variables at the bottom layer storage of smart contracts. The self-evolving Nebulas will be advantageous over other public blockchains in terms of developmental and survival potential. It also allows developers to respond faster to loopholes with upgrades and prevents huge losses caused by hacking.
    %我们认为,一个良态的区块链系统及其上的应用应该能够实现自我进化。在较少外部干涉的情况下,实现更快的计算、更强的系统、及更好的体验。我们将这种自我进化的能力称之为NF(Nebulas Force,星云原力)~(见\refsec{sec:nebulasforce})。在星云链的系统架构中,通过在区块结构上的良好设计,基础协议将会成为链上数据的一部分,并通过链上数据的追加实现基础协议的升级;对于星云链中的应用(智能合约),星云链通过在智能合约底层存储支持状态变量可跨合约访问的设计,完成智能合约的升级。具备自我升级进化能力的星云链,未来比其它的公有链具有更快的发展速度和更大的生存潜力,同时使得开发者面对漏洞,能够更快的响应 和升级,避免黑客事件给用户带来巨大的损失。

\item \textbf{The construction of blockchain application ecosystem}
%\item \textbf{区块链应用生态环境的建设}

We develop the PoD (see \refsec{sec:pod}) algorithm based on the devotion of accounts on Nebulas. This algorithm uses NR as the measure of value to identify the accounts with great devotion to the ecosystem, and grant them the right of bookkeeping on an equal basis to curb the monopoly in bookkeeping. It also integrates the economic penalties in PoS to prevent malicious damage to public blockchains, facilitating the freedom of ecosystem. Featured by faster consensus speed and stronger anti-cheat ability than PoS and PoI, PoD is a positive force for the development of blockchain ecosystem.
%在星云链中,我们提出了基于账户贡献度的PoD(见\refsec{sec:pod})算法,利用NR的价值尺度评估找出对生态贡献度较高的账户,平等地赋予记账资格,遏制记账权被垄断,并且融合PoS中的经济惩罚,防止公链被恶意破坏,为生态自由发展助力。既能保证较快的共识速度,又能比PoS和PoI更抗作弊,对区块链生态的发展有良好的促进作用。

We also develop the DIP (Developer Incentive Protocol) ~(see \refsec{sec:dip}) for smart contracts and DApp developers, which aims to reward smart contract or DApp developers for their great devotion to the community. The incentive is written into the blockchain by the bookkeeper. Based on the Nebulas Rank mechanism, Nebulas further includes a search engine ~(see \refsec{sec:search}) to help users better explore high-value applications in the blockchain.
%在星云链中,我们提出面向智能合约和DApp开发者的DIP(Developer Incentive Protocol, 开发者激励协议)~(见\refsec{sec:dip})。DIP的核心思想是对社区贡献度高的智能合约或DApp的开发者,给予他们相应的开发者激励。激励由记账人负责写入区块。
%基于Nebulas Rank机制,星云链进一步包含了搜索引擎~(见\refsec{sec:search}),以帮助用户更好的探索区块链中的高价值应用。
\end{itemize}

Since Ethereum is a successful public blockchain platform with an ecosystem of massive scale, Nebulas hopes to learn from Ethereum's excellent design, and make smart contract fully compatible with it, so that Ethereum-based applications can run on Nebulas with zero migration cost.
%考虑到以太坊已经有巨大的生态,是一个非常成功的公有区块链平台。星云链希望尽可能的借鉴以太坊等其他区块链系统的优秀设计,从智能合约编程上完全兼容以太坊,使得基于以太坊开发的产品能够零成本的迁移到星云链上。


Based on the above principles of design, we strive to build a blockchain operating system and a search engine based on the measure of value. This white paper describes in detail the technologies embedded in Nebulas. \refsec{sec:rank} explains the Nebulas Rank, a model of measure of value, and its algorithm; \refsec{sec:nebulasforce} describes the Nebulas Force, a self-evolving capability of Nebulas; \refsec{sec:dip}, \refsec{sec:pod}, \refsec{sec:search}, \refsec{sec:tools} are about Nebulas’ conception and design of the ecosystem for blockchain applications; and \refsec{sec:nascoin} discusses NAS, the token of Nebulas.
%基于上述设计原则,我们试图构建一个基于价值尺度的区块链操作系统及搜索引擎。本白皮书详细描述了星云链中关于技术的细节,其中\refsec{sec:rank}描述了一种可能的价值尺度及其算法Nebulas Rank,\refsec{sec:nebulasforce}描述了星云链的自我进化能力Nebulas Force, \refsec{sec:dip}、\refsec{sec:pod}、\refsec{sec:search}、\refsec{sec:tools}描述了星云链在建设区块链应用生态圈的的设计和构想,最后,\refsec{sec:nascoin}描述了星云链的代币NAS。
