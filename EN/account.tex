\section{星云链账号地址设计}

\subsection{地址校验码}
类似于比特币和以太坊,星云链的帐号也是由椭圆曲线算法作为其基础加密算法。
用户的私钥是随机生成的256位二进制数,通过椭圆曲线乘法生成64字节的公钥。
比特币和以太坊的地址都是由公钥通过确定性的哈希算法运算得到,不同的是:比特币地址有校验码设计,防止用户输错了几个字符而把比特币误发给其它用户,以太坊却没有校验码设计。

我们认为从用户的角度看,校验码的设计是合理的,因此星云链的地址也包含了校验码,具体计算方法如下:

\begin{verbatim}
  Data = sha3_256(Public Key)[-20:]
  CheckSum = sha3_256(Data)[0:4]
  Address = "0x" + Hex(Data + CheckSum)
\end{verbatim}

公钥的SHA3-256摘要的后20位字节作为地址的主要组成部分,对其再做一次SHA3-256摘要,取前4位字节做校验码,其相当于在以太坊的地址尾部加上4字节的校验码。

例如:
\begin{enumerate}
  \item Alice的以太坊钱包标准地址是 0xdf4d22611412132d3e9bd322f82e2940674ec1bc ;
  \item 最后星云链钱包地址是: 0xdf4d22611412132d3e9bd322f82e2940674ec1bc\textbf{03b20e40} 。
\end{enumerate}

以太坊地址包括0x前缀共42位字符,星云链地址增加了8位字符的校验码,共50位字符。

\subsection{扩展地址验证}
除了支持50字符的标准地址,为了用户转账的安全,我们还支持扩展地址。借鉴了传统银行转账的设计:转账时除了验证对方的银行卡账号,汇款人还必须输入对方的姓名,只有银行卡账号和姓名都匹配,转账才能正确进行。扩展地址的生成算法如下:

\begin{verbatim}
  Data = sha3_256(Public Key)[-20:]
  CheckSum = sha3_256(Data)[0:4]
  Address = "0x" + Hex(Data + CheckSum)

  ExtData = Utf8Bytes({Nickname or any string})
  ExtHash = sha3_256(Data + ExtData)[0:2]
  ExtAddress = Address + Hex(ExtHash)
\end{verbatim}

扩展地址在标准地址的尾部追加了2字节的扩展验证,共54位字符。加入扩展信息,使得在星云链钱包应用中可以增加另一个要素验证。例如:
\begin{enumerate}
	\item Alice的钱包标准地址是 0xdf4d22611412132d3e9bd322f82e2940674ec1bc03b20e40,加入昵称“alice”后的扩展地址是0xdf4d22611412132d3e9bd322f82e2940674ec1bc03b20e40\textbf{e345};
	\item Alice告诉Bob扩展地址 0xdf4d22611412132d3e9bd322f82e2940674ec1bc03b20e40e345 和昵称\textbf{alice};
	\item Bob在Wallet应用中,输入 0xdf4d22611412132d3e9bd322f82e2940674ec1bc03b20e40e345 和 alice;
	\item Wallet应用验证钱包地址的一致性,避免Bob错误的输成了其它人的帐号。
\end{enumerate}

标准地址和扩展地址均可以用于转账,我们的星云链钱包应用将会强制使用扩展地址,转账时需要提供对方的昵称,验证地址一致性,进一步加强安全校验。
